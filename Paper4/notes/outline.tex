{\large\textbf{Paper 4: Data integration for HIV contact network estimation: San Diego}}\\

\textbf{Purpose:}

\begin{itemize}
    \item Existing methods use general models for disease transmission over the network.  We hypothesize that we can gain efficiency by customizing the epidemic model to the disease of interest.
    \item Choice of epidemic model (4 stages of progression, on/off treatment) – \item justification and specification.
    \item Likelihood and MCMC to implement new epidemic model.
    \item Apply to San Diego data
    \item Notes:
    \begin{itemize}
        \item Include dyadic dependent terms: assortativity, clustering
    \end{itemize}
\end{itemize}

\textbf{Outline:}

\renewcommand{\labelenumi}{\arabic{enumi}.} 
\renewcommand{\labelenumii}{\arabic{enumi}.\arabic{enumii}}
\begin{enumerate}
    \item Terminology and notation

    \begin{enumerate}
        \item Data sources
        \item Contact Network
        \item Epidemic model
        \item Transmission network
    \end{enumerate}
    \item Model and Methods
    \begin{enumerate}
        \item Inferential model
        \item Data and Priors
        \item Model fitting
        \item Update G
        \item Update P
        \item Update E and I
        \item Update epidemic model parameters
    \end{enumerate}
\end{enumerate}

\renewcommand{\labelenumi}{\arabic{enumi}.} 
\renewcommand{\labelenumii}{\arabic{enumi}.\arabic{enumii}}
