The likelihood $L(\mathbf{R}| \beta, \theta, k, D, \G, \Pnet, \eta, \mathbf{T}, \mathbf{I})$ can be factored into 6 components: the contribution of contacts over which infections were transmitted, the contribution of contacts over which infections were not transmitted, 
the contribution of first transitions to each infectious phase, and the contribution of transitions from infectious phase to removal. These are:

\begin{align}
L_1 &= \prod_{(j,k) \in \mathscr{P}} \sum_{i=1}^{4} \beta_{i}\cdot \exp \{ -\beta_{i} \cdot \max (\min(I_{i+1}^{j}, I_{1}^{k})-I_{i}^{j},0) - \sum_{u=1}^{v} \max(\min(I_{1}^{k}, T_{i_{u,2}}^{j}) - T_{i_{u,1}} ,0) \}  \label{eq:l1} \\
L_2 &= \prod_{(j,k) \in \mathscr{G} \setminus \mathscr{P}} \prod_{i=1}^{4} \exp\{-\beta_{i}\cdot\max(\min(I_{i+1}^{j}, I_{1}^{k})-I_{i}^{j},0) - \sum_{u=1}^{v} \max(\min(I_{1}^{k}, T_{i_{u,2}}^{j}) - T_{i_{u,1}}^{j} ,0) \}  \label{eq:l2}\\
L_3 &= \frac{\prod_{l=1}^{m}{(I_{2}^{l}-I_{1}^{l})}^{I_{1}^{k}-1}\theta_I_{1}^{-m I_{1}^{k}}\cdot\exp\{{-\sum_{l=1}^{m}(I_{2}^{l}-I_{1}^{l})/\theta_{I_{1}}}\}}{\Gamma(I_{1}^{k})^m} \label{eq:l3} \\
L_4 &= \frac{\prod_{l=1}^{m}{(I_{3}^{l}-I_{2}^{l})}^{I_{2}^{k}-1}\theta_I_{2}^{-m I_{2}^{k}}\cdot\exp\{{-\sum_{l=1}^{m}(I_{3}^{l}-I_{2}^{l})/\theta_{I_{2}}}\}}{\Gamma(I_{2}^{k})^m} \label{eq:l4} \\
L_5 &= \frac{\prod_{l=1}^{m}{(I_{4}^{l}-I_{3}^{l})}^{I_{3}^{k}-1}\theta_I_{3}^{-m I_{3}^{k}}\cdot\exp\{{-\sum_{l=1}^{m}(I_{4}^{l}-I_{3}^{l})/\theta_{I_{3}}}\}}{\Gamma(I_{3}^{k})^m} \label{eq:l5} \\
L_6 &= \frac{\prod_{l=1}^{m}{(R^{l}-I_{4}^{l})}^{I_{4}^{k}-1}\theta_I_{4}^{-m I_{4}^{k}}\cdot\exp\{{-\sum_{l=1}^{m}(R^{l}-I_{4}^{l})/\theta_{I_{4}}}\}}{\Gamma(I_{4}^{k})^m} \label{eq:l6}
\end{align}

where $m$ is the total number ultimately infected, and $\mathbf{T}$ is a $u \times 2$ matrix of start and stop times for treatment spells. Superscript indicates the node to which that times corresponds. For notational convenience, $I_{5}$ is equivalent to $R$. We assume for the purposes of this notation that the pair $(j,k)$ is ordered such that $j$ is infected first (or the only infected if it is an S-I edge).

When a person $k$ is uninfected over the course of the epidemic, we take their transition times $(I^k, R^k)$ to be infinite. These combine to give us our likelihood of:

\begin{equation} \label{eq:lprod}
L = L_1 L_2 L_3 L_4 L_5 L_6
\end{equation}
